\chapter*{Введение}
\addcontentsline{toc}{chapter}{Введение}

Целью лаборатоной работы является изучение и реализация алгоритмов нахождения расстояний
Левенштейна и Дамерлау-Левештейна, а также получения навыка динамического программирования. Для её достижения необходимо выполнить следующие задачи:
\begin{itemize}
	\item Изучение алгоритмов Левенштейна и Дамерлау-Левештейна
	\item Разработать данные алгоритмы
	\item Применение методов динамического программирования для реализации алгоритмов
	\item Выполнить тестирование реализации алгоритмов методом черного ящика
	\item Провести сравнительный анализ этих алгоритмов по затратам памяти и процессорному выполнению времени на основе экспериментальных данных
\end{itemize}