\chapter{Аналитический раздел}

\section{Расстояние Левенштейна}
Расстояние Левенштейна (редакторское расстояние) между двумя строками - минимальное количество операций вставки, удаления и замены, необходимых для превращения одной строки в другую.

При преобразовании одной строки в другую используются следующие операции:

\begin{enumerate}[a)]
	\item I (insert) - вставка;
	\item D (delete) - удаление;
	\item R (replace) - замена; 
\end{enumerate}
Будем считать, что стоимость каждой из этих операции (штраф) равна единице.

Введем еще одну операцию M (match) - совпадение. Её стоимость будет равна нулю.

Необходимо найти последовательность замен с минимальным суммарным штрафом.

\section{Реккурентный алгоритм}
Расстояние между двумя строками s1 и s2 рассчитывается по реккуретной формуле \ref{eq:1}:
$$
D(s1[1..i], s2[1..j]) = 
\begin{cases}
	0, \quad \phantom{\infty}\text{i=0, j=0}\\
	j, \quad \phantom{\infty}\text{i=0, j>0}\\
	i, \quad \phantom{\infty}\text{i>0, j=0}\\
	\min \begin{cases}
		D(s1[1..i], s2[1..j-1]) + 1\\
		D(s1[1..i-1], s2[1..j]) + 1\\
		D(s1[1..i-1], s2[1..j-1]) + \\f(s1, s2)
	\end{cases}
\end{cases}
\label{eq:1}
\eqno(1.2.1)
$$
Функция f(s1, s2) определяется следующим образом:
$$
f(s1, s2) =
\begin{cases}
	0, \text{  s1=s2}\\
	1, \text{  иначе}
\end{cases}
\eqno(1.2.2)
$$

\section{Матрица расстояний}
Реализация алгоритма по формуле (1.2.1) при больших значениях i,j, оказывается менее эффективной по времени ввиду того, что приходится вычислять промежуточные результаты неоднократно. Для оптимизации нахождения расстояния Левенштейна необходимо использовать матрицу стоимостей для хранения этих промежуточных значений. В таком случае алгоритм представляет собой построчное заполнение матрицы значениями D(i, j) 

\section{Использование двух строк}
Модификацией использования матрицы является использование только двух строк этой матрицы, в которых хранятся промежуточные значения. После выполнения вычислений выполняется обмен значений этих двух строк. Алгоритм продолжает работать и перезаписывать значения только второй строки.

\section{Рекурсивный алгоритм с кэшем в форме матрицы}
При помощи матрицы можно выполнить оптимизацию рекурсивного алгоритма заполнения. Основная идея такого подхода заключается в том, что при каждом рекурсивном вызове алгоритма выполняется заполнение матрицы стоимостей. Главное отличие данного метода от того, что был описан в разделе (1.3) - начальная инициализация матрицы значением $\infty$. Если рекурсивный алгоритм выполняет вычисления для данных, которые не были обработаны, значение результата минимального расстояния для данного вызова заносится в матрицу. Если рекурсивный вызов уже обрабатывался (ячейка матрицы была заполнена), то алгоритм не выполняет вычислений, а сразу переходит к следующему шагу.

\section{Расстояние Дамерау-Левенштейна}
Расстояние Дамерау-Левенштейна является модификацией расстояния Левенштейна, которая задействует еще одну редакторскую операцию - транспозицию T
(transposition. Она выполняет обмен соседних символов в слове.

Дамерау показал, что 80\% человеческих ошибок при наборе текстов является перестановка соседних символов, пропуск символа, добавление нового символа или ошибочный символ\cite{Damerau_Levenshtein}. Таким образом, расстояние Дамерау-Левенштейна часто используется в редакторских программах для проверки правописания.
Это расстояние может быть вычислено по следующей реккуретной формуле:

$$
D(s1[1..i], s2[1..j]) = 
\begin{cases}
	0, \quad \phantom{\infty}\text{i=0, j=0}\\
	j, \quad \phantom{\infty}\text{i=0, j>0}\\
	i, \quad \phantom{\infty}\text{i>0, j=0}\\
	\min \begin{cases}
		D(s1[1..i], s2[1..j-1]) + 1\\
		D(s1[1..i-1], s2[1..j]) + 1\\
		D(s1[1..i-1], s2[1..j-1]) + f(s1, s2)\\
		D(s1[1..i-1], s2[1..j-1]) + 1, \\
		\text{   i,j > 1$,a_{i}=b_{j-1},$ $a_{i-1}=b_{j}$}\\
		{\infty},\text{      иначе}\\
	\end{cases}\\
	
\end{cases}
\label{eq}
\eqno(1.5.1)
$$

\section*{Вывод}
В данном разделе были рассмотрены основные способы нахождения редакторского расстояния между двумя строками. Формулы для нахождения расстояния Левенштейна и Дамерау-Левенштейна задаются реккуретно, следовательно, алгоритмы могут быть реализованы как рекурсивно, так и итерационно.