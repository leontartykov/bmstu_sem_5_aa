\chapter*{Введение}
\addcontentsline{toc}{chapter}{Введение}
Матричное умножение лежит в основе нейронных сетей. Большинство операций при обучении нейронной сети требуют некоторой формы умножения матриц. Для этого требуется высокая скорость вычислений.

Стандартный алгоритм умножения матриц на больших данных, исчисляемых миллиардами, выполняет вычисления не самым быстрым способом. Существуют различные оптимизации. Одной из них является алгоритм Винограда, который позволяет сократить время вычислений. На практике алгоритм Копперсмита—Винограда не используется, так как он имеет очень большую константу пропорциональности и начинает выигрывать в быстродействии у других известных алгоритмов только для тех матриц, размер которых превышает память современных компьютеров.

Целью лаборатоной работы является изучение и реализация алгоритмов умножения матриц. Для её достижения необходимо выполнить следующие задачи:
\begin{itemize}
	\item изучить алгоритм умножения матриц стандартным способом и по Винограду;
	\item разработать данные алгоритмы;
	\item оптимизировать алгоритм Винограда;
	\item выполнить тестирование реализации алгоритмов методом черного ящика;
	\item провести сравнительный анализ этих алгоритмов по затратам памяти и процессорному выполнению времени на основе экспериментальных данных.
\end{itemize}