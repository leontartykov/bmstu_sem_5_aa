\chapter{Исследовательский раздел}
\section{Технические характеристики}
Технические характеристии устройства, на котором выполнялось тестирование:
\begin{itemize}
	\item операционная система: Windows 10 Pro;
	\item память: 8 GiB;
	\item процессор: Intel(R) Core(TM) i5-8265U CPU @ 1.60GHz   1.80 GHz.
\end{itemize}
Тестирование проводилось на ноутбуке, который был подключен к сети питания. Во время проведения тестирования ноутбук был нагружен только встроенными приложениями окружения, самим окружением и системой тестирования.

\section{Временные харастеристики выполнения}
Ниже был проведен анализ времени работы алгоритмов. Исходными данными является квадратная матрица целых чисел. Единичные замеры выдадут крайне маленький результат, поэтому  проведем работу каждого алгоритма n = 5 раз и поделим на число n. Получим среднее значение работы каждого из алгоритмов. 

Выполним анализ для случая, когда размер матриц целых чисел имеет нечетный размер \{101, 201,..., 1001\}. Результат приведен на рис \ref{fg:6_1}.

\begin{figure}[H]
	\centering
	\begin{tikzpicture}
		\begin{axis}
			[grid = major,
			 xlabel = Размер матрицы,
			 ylabel = {Время, c},
			 ymin = 0,
			 width = 0.95\textwidth,
			 height=0.3\textheight,
			 legend style={at={(0.5,-0.2)},anchor=north},
			 xmajorgrids=true]
		\addplot table{data/file_4.txt};
		\addplot table{data/file_5.txt};
		\addplot table{data/file_6.txt};
		\legend{
			Стандартный алгоритм,
			Алгоритм Винограда,
			Оптимизированный алгоритм Винограда
		};
		\end{axis}
	\end{tikzpicture}
	\caption{График зависимости времени работы алгоритмов при чётных размерностях матриц} 
	\label{fg:6_1}
\end{figure}

Как видно из результатов, алгоритм умножения матриц по Винограду работает медленнее стандартного приблизительно на 12\% на размерах \{700, ..., 1000\}. Оптимизированная версия алгоритма Винограда выполняет вычисления быстрее стандартного примерно на 47\% и выигрывает по скорости у обычного алгоритма умножения по Винограду в среднем на 63\% на размерах \{700, ..., 1000\}.


Выполним анализ для случая, когда размер матриц целых чисел имеет четный размер \{100, 200,..., 1000\}. Результат приведен на рис \ref{fg:6_2}.
\begin{figure}[H]
	\centering
	\begin{tikzpicture}
		\begin{axis}
			[grid = major,
			xlabel = Размер матрицы,
			ylabel = {Время, c},
			ymin = 0,
			width = 0.95\textwidth,
			height=0.3\textheight,
			legend style={at={(0.5,-0.2)},anchor=north},
			xmajorgrids=true]
				
			\addplot table{data/file_1.txt};
			\addplot table{data/file_2.txt};
			\addplot table{data/file_3.txt};
			\legend{
				Стандартный алгоритм,
				Алгоритм Винограда,
				Оптимизированный алгоритм Винограда
			};
			\end{axis}
		\end{tikzpicture}
		\caption{График зависимости времени работы алгоритмов при чётных размерностях матриц} 
		\label{fg:6_2}
	\end{figure}
Как видно из результатов, все приведенные алгоритмы работают дольше, чем в ситуации, кога размер матриц четный. Алгоритм умножения матриц по Винограду работает медленнее стандартного приблизительно на 9\%. Оптимизированная версия алгоритма Винограда выполняет вычисления быстрее стандартного на размерах \{700, ..., 1000\} в среднем на 40\% и выигрывает по скорости у обычного алгоритма умножения по Винограду на 51\% на размерах \{700, ..., 1000\}.

\section{Вывод}
Экспериментально была подтвержена трудоемкость алгоритмов умножения матриц, описанная в разделах 2.2.1, 2.2.2, 2.2.3. Время выполнения, описанное в разделе 4.2., каждого из этих методов при нечетных размерностях матриц больше времения выполнения тех же методов в случае, когда размерность матриц четная. Это связано с дополнительными операциями обработки, указанных в схемах алгоритма раздела 2.3.