\chapter*{Введение}
\addcontentsline{toc}{chapter}{Введение}
При наборе текста часто возникает проблема, связанная с опечатками. Необходимы средства, которые позволили бы быстро исправлять эти ошибки.

Для решения подобных задач в прикладной лингвистике выделяется такое направление, как компьютерная лингвистика, в которой разрабатываются и используются компьютерные программы, необходимые для исследования языка и моделирования функционирования языка в тех или иных условиях\cite{book1}.

Одним из первых, кто занялся такой задачей, был советский ученый В.И.Левенштейн\cite{Levenshtein}. Алгоритм полученного решения связали с его именем. Расстояние Левенштейна - метрика, измеряющая разность двух строк, определяемая в количестве редакторских операций (вставка, удаление, замена), требуемых для преобразования одной последовательности символов в другую. Модификацией данного алгоритма является расстояние Дамерау-Левенштейна, которая добавляет транспозицию, обмен двух соседних символов, к редакторским операциям. Разработанные алгоритмы нашли применение не только в компьютерной лингвистике, но и в биоинформатике для определения схожести разных участков ДНК и РНК.

Целью лаборатоной работы является изучение и реализация алгоритмов нахождения расстояний Левенштейна и Дамерлау-Левештейна, а также получения навыка динамического программирования. Для её достижения необходимо выполнить следующие задачи:
\begin{itemize}
	\item изучение алгоритмов Левенштейна и Дамерлау-Левештейна;
	\item разработать данные алгоритмы;
	\item применение методов динамического программирования для реализации алгоритмов;
	\item выполнить тестирование реализации алгоритмов методом черного ящика;
	\item провести сравнительный анализ этих алгоритмов по затратам памяти и процессорному выполнению времени на основе экспериментальных данных;
\end{itemize}