\chapter{Технологический раздел}
\section{Требования к программному обеспечению}
Программа должна отвечать следующим требованиям:
\begin{itemize}
	\item на вход программе подается массив целых чисел и размерность этого массива;
	\item на выход программа выдает отсортированный массив целых чисел.
\end{itemize}

\section{Выбор средств реализации}
Для реализации алгоритмов в данной лабораторной работе был выбран язык программирования Python 3.9.7\cite{python3}. Он является кроссплатформенным. Имеется опыт разработки на этом языке. В качестве среды разработки был использован Visual Studio Code\cite{vs}, так как в нем можно работать как на операционной системе Windows, так и на дистрибутивах Linux. При замере процессорного времени был использован модуль time\cite{time}. Замеры используемой памяти и число вызовов рекурсии проводились при помощи модуля cProfiler\cite{cprofile}.
\section{Листинги программ}
Ниже представлены листинги разработанных алгоритмов Левенштейна и Дамерау-Левенштейна.

\begin{lstlisting}[label=some-code,caption=Программный код сортировки пузырьком]
def bubble_sort(array: list[int], count: int) -> list[int]:
	for i in range(count):
		for j in range(count - i - 1):
			if array[j] > array[j + 1]:
				array[j], array[j + 1] = array[j + 1], array[j]
\end{lstlisting}

\begin{lstlisting}[label=some-code,caption=Программный код сортировки вставками]
def insert_sort(array: list[int], count: int) -> list[int]:
	for i in range(1, count):
		select_item = array[i]
		j = i - 1
		while j >= 0 and select_item < array[j]:
			array[j+1] = array[j]
			j -= 1
		array[j + 1] = select_item
\end{lstlisting}

\begin{lstlisting}[label=some-code,caption=Программный код сортировки методом Шелла]
def shell_sort(array: list[int], count:int) -> list[int]:
	distance = count // 2
	while distance > 0:
		for i in range(count - distance):
		j = i
		while j >= 0 and array[j] > array[j + distance]:
			array[j + distance], array[j] = array[j], array[j + distance]
			j -= 1
		distance //= 2
\end{lstlisting}

\section{Тестирование}
Для тестирования используется метод черного ящика. В данном разделе приведена таблица \ref{table:ref1}, в которой указаны классы эквивалентностей тестов: \\

\begin{table}[ht!]
	\centering
	\captionsetup{singlelinecheck = false, justification=raggedleft}
	\caption{Таблица тестов}
	\label{table:ref1}
	\begin{tabular}{|c|c|c|c|c|c|}
		\hline
		\multirow{3}{*}{№} & \multirow{3}{*}{Описание теста} & \multirow{3}{*}{Слово 1}  &  \multirow{3}{*}{Слово 2}   & \multicolumn{2}{|c|}{Алгоритм}\\ \cline{5-6}
		&                &          &            &\multirow{2}{*}{Левенштейн}   &Дамерау-	\\ 
		&                &          &            &             &Левенштейн       	        \\ \hline
		1& Пустые строки  &  ''      &    ''      &   0         &  0 						\\ \hline
		\multirow{2}{*}{2}& \multirow{2}{*}{Нет повторяющихся} & \multirow{2}{*}{deepcopy} & \multirow{2}{*}{раздел} & \multirow{2}{*}{8}   &  \multirow{2}{*}{8}                      
		\\
		 & символов        &          &            &             &
		 \\ \hline
		3& Инверсия строк & insert   &tresni      &   6         &  6                       \\ \hline
		4& Два соседних символа       & heart    & heatr  & 2   &  1                       \\ \hline
		5& Одинаковые строки          & таблица  & таблица& 0   &  0						\\ \hline
		\multirow{2}{*}{6}& \multirow{2}{*}{Одна строка} &\multirow{2}{*}{город} &\multirow{2}{*}{горо} & \multirow{2}{*}{1} & \multirow{2}{*}{1} \\  
		& меньше другой   &           &           &      &\\ \hline
	\end{tabular}
\end{table}

\section*{Вывод}
В данном разделе был выбран язык программирования, среда разработки. Реализованы функции, описанные в аналитическом разделе, и проведено их тестирование методом черного ящика по таблице \ref{table:ref1}. 

