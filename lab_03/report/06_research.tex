\chapter{Исследовательский раздел}
\section{Технические характеристики}
Технические характеристики устройства, на котором выполнялось тестирование:
\begin{itemize}
	\item операционная система: Windows 10 Pro;
	\item память: 8 GiB;
	\item процессор: Intel(R) Core(TM) i5-8265U CPU @ 1.60GHz   1.80 GHz.
\end{itemize}
Тестирование проводилось на ноутбуке, который был подключен к сети питания. Во время проведения тестирования ноутбук был нагружен только встроенными приложениями окружения, самим окружением и системой тестирования.

\section{Временные харастеристики выполнения}
Ниже был проведен анализ времени работы алгоритмов. Исходными данными является массив. Единичные замеры выдадут крайне маленький результат, поэтому  проведем работу каждого алгоритма n = 100 раз и поделим на число n. Получим среднее значение работы каждого из алгоритмов. 

Выполним анализ для случая, когда массив целых чисел упорядочен по возрастанию. Результат приведен на рис \ref{fg:ref1}.

\begin{figure}[H]
	\centering
	\begin{tikzpicture}
		\begin{axis}
			[grid = major,
			xlabel = Размер матрицы,
			ylabel = {Время, c},
			ymin = 0,
			width = 0.95\textwidth,
			height=0.3\textheight,
			legend style={at={(0.5,-0.2)},anchor=north},
			xmajorgrids=true]
			\addplot table{data/best_bubble_time.txt};
			\addplot table{data/best_insert_time.txt};
			\addplot table{data/best_shell_time.txt};
			\legend{
				Пузырек,
				Простые вставки,
				Шелл
			};
		\end{axis}
	\end{tikzpicture}
	\caption{График зависимости времени работы алгоритмов сортировки для лучшего случая.} 
	\label{fg:ref1}
\end{figure} 

Скорость выполнения алгоритма сортировки пузырьком отличается по сравнению с алгоритмом простыми вставками - 22\% для размерности 200, 77\% для 600, 201\% - для 1000. Алгоритм Шелла выигрывает у простых вставок по времени выполнения приблизительно на 13\% для размерности массива 500, 20\% для 800 и 21\% для 1000.

Выполним анализ для случая, когда массив целых чисел имеет случайные значения. Результат приведен на рис. \ref{fg:ref2}.
\begin{figure}[H]
	\centering
	\begin{tikzpicture}
		\begin{axis}
			[grid = major,
			xlabel = Размер массива,
			ylabel = {Время, c},
			ymin = 0,
			width = 0.95\textwidth,
			height=0.3\textheight,
			legend style={at={(0.5,-0.2)},anchor=north},
			xmajorgrids=true]
			\addplot table{data/usual_bubble_time.txt};
			\addplot table{data/usual_insert_time.txt};
			\addplot table{data/usual_shell_time.txt};
			\legend{
				Пузырек,
				Простые вставки,
				Шелл
			};
		\end{axis}
	\end{tikzpicture}
	\caption{График зависимости времени работы алгоритмов сортировки для обычного случая.} 
	\label{fg:ref2}
\end{figure}
Как видно из результатов алгоритм пузырька вновь уступает по скорости выполнения. На размерности массива 600 разница между двумя этими алгоритмами составляет приблизительно 183\%, 800 - 204\%, 1000 - 205\%. Метод простых вставок уступает по скорости выполнения сортировки методу Шелла на тех же размерностях в среднем на 309\%, 261\%, 321\% соответственно. 

Выполним анализ для случая, когда массив целых чисел отсортирован в обратном порядке. Результат приведен на рис. \ref{fg:ref3}.
\begin{figure}[H]
	\centering
	\begin{tikzpicture}
		\begin{axis}
			[grid = major,
			xlabel = Размер массива,
			ylabel = {Время, c},
			ymin = 0,
			width = 0.95\textwidth,
			height=0.3\textheight,
			legend style={at={(0.5,-0.2)},anchor=north},
			xmajorgrids=true]
			\addplot table{data/worst_bubble_time.txt};
			\addplot table{data/worst_insert_time.txt};
			\addplot table{data/worst_shell_time.txt};
			\legend{
				Пузырек,
				Простые вставки,
				Шелл
			};
		\end{axis}
	\end{tikzpicture}
	\caption{График зависимости времени работы алгоритмов сортировки для худшего случая.} 
	\label{fg:ref3}
\end{figure}

В данном эксперименте наблюдается следующая ситуация: сортировка пузырьком уступает простым вставкам в скорости выполнения на размерностях массива 400, 600, 800, 1000 приблизительно на 130\%, 137\%, 150\%, 139\% соответственно. Алгоритм сортировки простыми вставками уступает методу Шелла для тех же размерностей массива приблизительно на 2489\%, 2500\%, 3142\%, 3866\% соответственно. 

\section{Вывод}
Скорость выполнения алгоритма методом пузырька уступает двум остальным алгоритмам во всех случаях (лучший, обычный, худший) за счет обмена элементов массива друг с другом. Экспериментально подтвердилась трудоемкость выполнения каждого из алгоритмов. 