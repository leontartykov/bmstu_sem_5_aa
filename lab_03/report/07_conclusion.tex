\chapter*{Заключение}
\addcontentsline{toc}{chapter}{Заключение}
В ходе выполнения лабораторной работы были рассмотрены алгоритмы нахождения расстояния Левенштейна и Дамерау-Левенштейна. Были выполнено описание каждого из этих алгоритмов, приведены соотвествующие математические расчёты. Были получены навыки динамического программирования, а также реализованны данные алгоритмы. При тестировании каждого их них и анализе временных характеристик и объема потребляемой памяти можно сделать следующие выводы: выбор алгоритма Дамерау-Левенштейна является наиболее оптимальным решением ввиду того, что чаще всего необходимо исправлять ошибки, связанные с обменом двух соседних символов. В ином случае этот алгоритм является проигрышным как по времени, так и по памяти в сравнении с различными реализациями алгоритма Левенштейна. Самым быстрым среди данной группы алгоритмов по времени является рекурсивный алгоритм Левенштейн с кэшем в виде матрицы; но он достаточно много использует памяти за счет большого числа вызовов. Поэтому в иных ситуациях, не связанных с транспозицей, следует использовать итеративный алгоритм.
  