\chapter*{Заключение}
\addcontentsline{toc}{chapter}{Заключение}
В ходе выполнения лабораторной работы были рассмотрены алгоритмы сортировки. Были выполнено описание каждого из этих алгоритмов, приведены соотвествующие математические расчёты. Была рассчитана трудоемкость каждого из этого алгоритмов. При исследовании трудоемкости и времени выполнения алгоритмов сортировок можно сделать следующие выводы: временная сложность алгоритма не эквивалентна времени выполнения. Это можно хорошо увидеть на исследовании скорости выполнения алгоритма пузырька и метода Шелла. В худшем случае, когда на вход алгоритму подается массив целых чисел, отсортированный в обратном порядке, метод пузырька и метод Шелла имеет одинаковую вычислительную сложности, но второй работает значительно быстрее по сравнению с первым алгоритмом. Самым медленным из приведенных алгоритмов является алгоритм сортировки пузырьком, самым быстрым - сортировка простыми вставками. 
  