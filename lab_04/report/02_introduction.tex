\chapter*{Введение}
\addcontentsline{toc}{chapter}{Введение}
При выполнении множества задач необходимо задействовать эффективно как можно много ресурсов системы. Для этого необходимо увеличивать скорость выполнения программ. В настоящее время по определенным техническим причинам стало невозможным увеличивать тактовую частоту процессора. Однако есть другой способ увеличения производительности – размещение нескольких ядер в процессоре, но это требует другого подхода в программировании.

Параллельное программирование - новый подход в технологии разработки программного обеспечения, которое основывается на понятии "поток"\cite{parallel}. Поток - часть кода программы, которая может выполняться параллельно с другими частями кода программы. Многопоточность - способность центрального процессора или одного ядра в многоядерном процессоре одновременно выполнять несколько потоков.

Целью лаборатоной работы является изучение и реализация параллельного программирования для решения поиска минимального элемента в матрице. Для её достижения необходимо выполнить следующие задачи:
\begin{itemize}
	\item исследовать подходы паралелльного программирования;
	\item привести схемы алгоритмов последовательного и паралелльного поиска минимального элемента матрицы;
	\item описать используемые структуры данных;
	\item выполнить тестирование реализации алгоритмов методом черного ящика;
	\item провести сравнительный анализ этих алгоритмов по процессорному выполнению времени на основе экспериментальных данных.
\end{itemize}