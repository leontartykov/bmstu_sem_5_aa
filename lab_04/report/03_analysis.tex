\chapter{Аналитический раздел}

В данном разделе рассматривается описание общей задачи, принцип распределения задач.

\section{Определение матрицы}
Матрицей размера $m \times n$ называется прямоугольная таблица элементов некоторого множества 
(например, чисел или функций), имеющая m строк и n столбцов \cite{angem}.
Элементы $a_{ij}$ , из которых составлена матрица, называются элементами матрицы.
Условимся, что первый индекс $i$ элемента $a_{ij}$ a
соответствует номеру строки, второй индекс $j$ – номеру столбца, в котором расположен элемент $a_{ij}$.
Матрица может быть записана по формуле (\ref{eq:1_1}).

\begin{equation}
	A = \left(
	\begin{array}{cccc}
		a_{11} & a_{12} & \ldots & a_{1n} \\
		a_{21} & a_{22} & \ldots & a_{2n} \\
		\vdots & \vdots & \ddots & \vdots \\
		a_{n1} & a_{n2} & \ldots & a_{nn}
	\end{array}
	\right)
	\label{eq:1_1}
\end{equation}

\section{Алгоритм поиска минимума в матрице.}
Для решения нахождения поиска минимума в матрице необходимо выполнить следующие действия: для каждой строки матрицы $A[i]$ нужно выполнить вычисления локального максимума по формуле (\ref{eq:1_2}).

\begin{equation}
	loc_{min} = min(a_{i0}, ... , a_{in})
	\label{eq:1_2}
\end{equation}
Затем полученные каждые локальные минимумы сравниваются друг с другом. Таким образом, будет выполнен поиск глобального минимума в матрице.

\section{Параллельная реализация алгоритма}
В каждой строке матрицы выполняются схожие вычисления по нахождению в ней минимального элемента. Эти действия являются независимыми, причем строка матрицы не изменяется, поэтому для параллельного вычисления целесообразно разделить данную задачу между потоками. Каждый поток будет выполняться с необходимым объемом данных.

\section{Вывод}
В данном разделе были описаны необходимая задача и принцип её разделения. На вход программному обеспечению подается матрица; на выход программа должна выдать результат - значение максимального элемента в матрице. Все данные, подаваемые на вход, являются корректными. 