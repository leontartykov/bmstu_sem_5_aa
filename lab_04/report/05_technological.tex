\chapter{Технологический раздел}
В данном разделе приведены средства реализации и листинги кода.

\section{Выбор средств реализации}
Для реализации алгоритмов в данной лабораторной работе был выбран язык программирования C++ \cite{c++}. Данный выбор обусловен тем, что он содержит нативный код, что необходимо для уменьшения времени выполнения алгоритмов с помощью распараллеливания. Выбранным механизмом работы с потоками предоставляет библиотека pthread\cite{pthreads}. POSIX определяет набор интерфейсов для программирования потоков. В качестве среды разработки был использован Visual Studio Code\cite{vs}, так как в нем присутсвует поддержка практически всех языков программирования. Выполнение замера процессорного времени осуществлено при помощи библиотеки chrono \cite{chrono}.

\section{Листинги программ}
Ниже представлены листинги разработанных алгоритмов умножения матриц.
\newpage
\begin{lstlisting}[label = ordinary_min, caption=Программный код алгоритма нахождения минимума в матрице (последовательная реализация).]
	int find_matrix_min_value(int **matrix, int size_matrix)
	{
		int min = matrix[0][0], local_min = matrix[0][0];
		for (int i = 0; i < size_matrix; i++)
		{
			for (int j = 0; j < size_matrix; j++)
			{
				if (matrix[i][j] < local_min)
				local_min = matrix[i][j];
			}
			if (local_min < min)
			min = local_min;
		}
		return min;
	}
\end{lstlisting}

\begin{lstlisting}[label=vinograd_mult,caption=Программный код алгоритма нахождения минимума в матрице (параллельная реализация).]
	void *find_matrix_min_value_parallel(void *args)
	{
		pthread_args_t *mult_args = (pthread_args_t *)args;
		int row_start = mult_args->thread_id * (mult_args->matrix_size / mult_args->count_threads);
		int row_end = (mult_args->thread_id + 1)* (mult_args->matrix_size / mult_args->count_threads);
		int **matrix = mult_args->args->matrix;
		mult_args->local_min = matrix[0][0];
		
		for (int i = row_start; i < row_end; i++)
		{
			for (int j = 0; j < mult_args->matrix_size; j++)
			{
				if (matrix[i][j] < mult_args->local_min)
				mult_args->local_min = matrix[i][j];
			}
		}
		
		return NULL;
	}
\end{lstlisting}

\section{Вспомогательные модули}
На листингах представлены программные модули, которые используются в данных функциях:
\begin{lstlisting}[label=lst:struct_matrix,caption=Программный код класса по работе с матрицами.]	
	for (int k = 0; k < count_threads; k++){
		pthread_create(threads + k, NULL, find_matrix_min_value_parallel, args + k);
	}
	
	for (int k = 0; k < count_threads; k++){
		pthread_join(threads[k], NULL);
	}
	int global_min = args[0].local_min;
	for (int i = 1; i < count_threads; i++){
		if (args[i].local_min < global_min)
		global_min = args[i].local_min;
	}
\end{lstlisting}

\section{Тестирование}
Для тестирования используется метод черного ящика. В данном разделе приведена таблица \ref{table:ref1}, в которой указаны классы эквивалентностей тестов: \\

\begin{table}[ht!]
	\centering
	\captionsetup{singlelinecheck = false, justification=raggedright}
	\caption{Таблица тестов}
	\label{table:ref1}
	\begin{tabular}{|c|c|c|c|}
		\hline
		№ &Описание теста & Матрица  &    Ожидаемый результат\\\hline
		1& Квадратная матрица  & $\begin{pmatrix}6 & 9 & 8\\0 & 3 & 6\\4 & 9 & 5\end{pmatrix}$ & 0
		\\ \hline
		2& Прямоугольная матрица	  & $\begin{pmatrix}2 & 3 & -2\\3 & 9 & 2\end{pmatrix}$ & -2
		\\ \hline
		3& Несколько минимумов		& $\begin{pmatrix}5 & 1\\1 & 4\\6 & 4\end{pmatrix}$ & 1
		\\ \hline
	\end{tabular}
\end{table}
	Для последовательной и параллельной реализации алгоритма поиска минимума в матрице все тесты пройдены успешно.
\section{Вывод}
В данном разделе был выбран язык программирования, среда разработки, описан механизм работы с потоками. Выполнена реализация последовательного и параллельного алгоритма. Было успешно проведено их тестирование методом черного ящика по таблице \ref{table:ref1}. 

