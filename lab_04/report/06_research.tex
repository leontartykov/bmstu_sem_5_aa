\chapter{Исследовательский раздел}
В данном разделе приведены технические характеристики устройства, а также таблица сравнения скорости выполнения последовательной и параллельной реализации алгоритма.
 
\section{Технические характеристики}
Технические характеристики устройства, на котором выполнялось тестирование:
\begin{itemize}
	\item операционная система: Ubuntu 20.04;
	\item память: 8 GiB;
	\item процессор: Intel(R) Core(TM) i5-8265U CPU @ 1.60GHz   1.80 GHz;
	\item количество ядер: 4;
	\item количество логических ядер: 8.
\end{itemize}
Тестирование проводилось на ноутбуке, который был подключен к сети питания. Во время проведения тестирования ноутбук был нагружен только встроенными приложениями окружения, самим окружением и системой тестирования.

\section{Временные харастеристики выполнения}
Ниже был проведен анализ времени работы алгоритма для последовательной и параллельной реализации. Проведем работу каждого из них n = 10 раз и поделим на число n. Получим среднее значение работы каждого из алгоритмов. Для параллельного алгоритма выполнение вычислений производилось на $M \in \{1, 2, 4, 8, ... , 4 \cdot L\}$ потках, где L - число логических ядер данной ЭВМ. Тестирование проведено для квадратных матриц целых чисел.

Выполним анализ скорости выполнения последовательной и параллельной реализации алгоритма поиска минимума в матрице. Результат приведен в таблицах \ref{table:4_1}.

\begin{table}[ht!]
	\centering
	\captionsetup{singlelinecheck = false, justification=raggedright}
	\caption{Таблица времени реализации последовательного и параллельного алгоритмов (в секундах).}
	\label{table:4_1}
	\begin{tabular}{|c|c|c|c|c|}
		\hline
		\multirow{2}{*}{Размер}& \multirow{2}{*}{Последовательный} & \multicolumn{3}{c|}{Количество потоков} \\ \cline{3-5}
		матрицы& &1 поток & 2 потока &4 потока \\ \hline
		100 & 2.75149e-05 & 7.89085e-05 & 8.27494e-05 & 9.9439e-05 \\ \hline
		250 & 0.00017374  & 0.000256578 & 0.000144029 & 0.000149204 \\ \hline
		500 & 0.000787548 & 0.000863721 & 0.000454607 & 0.000420068 \\ \hline
		1000 & 0.00316544 & 0.00327061  & 0.00177363  & 0.00145989 \\ \hline
		2000 & 0.0135465  & 0.0129128   & 0.00711915  & 0.0055497 \\ \hline
		3500 & 0.0385047  & 0.04707425  & 0.0213427   & 0.0202112 \\ \hline
		5000 & 0.0768607  & 0.0804154   & 0.047094    & 0.0342458 \\\hline
	\end{tabular}
\end{table} 

\begin{table}[ht!]
	\centering
	\captionsetup{singlelinecheck = false, justification=raggedright}
	\caption{Таблица времени реализации последовательного и параллельного алгоритмов (в секундах).}
	\label{table:4_2}
	\begin{tabular}{|c|c|c|c|c|}
		\hline
		\multirow{2}{*}{Размер}& \multirow{2}{*}{Последовательный} & \multicolumn{3}{c|}{Количество потоков} \\ \cline{3-5}
		матрицы& &8 потоков & 16 потоков &32 потока \\ \hline
		100 & 2.75149e-05 & 0.0019957   & 0.00431042  & 0.000867361 \\ \hline
		250 & 0.00017374  & 0.000243679 & 0.000460978 & 0.0010327 \\ \hline
		500 & 0.000787548 & 0.000504962 & 0.000642802 & 0.000964574 \\ \hline
		1000 & 0.00316544 & 0.00165852  & 0.0016858   & 0.00174856 \\ \hline
		2000 & 0.0135465  & 0.00564542  & 0.00490783  & 0.00503544 \\ \hline
		3500 & 0.0385047  & 0.0182714   & 0.0142113   & 0.0137701 \\ \hline
		5000 & 0.0768607  & 0.0296331   & 0.0280696   & 0.0267538 \\\hline
	\end{tabular}
\end{table} 

\section{Вывод}
По результатам проведенного эксперимента можно сделать вывод о скорости выполнения каждой из реализацией и целесообразности создания потоков: для любых размеров матриц не имеет смысла создавать параллельную реализацию для одного потока, так как для этого требуются дополнительные временные затраты на его создание. Скорость выполнения при использовании двух потоков меньше на 87\% скорости последовательной реализации на размерах 100, 250. Также для этих размеров квадратной матрицы время выполнения для 8 потоков, что соответствует числу логических процессоров данной ЭВМ, меньше по сравнению с временем для 16 и 32 на 45\%. Для размерностей, начиная с 2000, эта разница составляет уже 2,5 раза.