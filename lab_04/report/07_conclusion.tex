\chapter*{Заключение}
\addcontentsline{toc}{chapter}{Заключение}
В данной лабораторной работе были рассмотрены основополагающие материалы, которые потребовались при параллельной и однопоточной реализации алгоритма поиска минимального элемента в матрице. Были рассмотрены схемы последовательной реализации и многопоточной, произведен их сравнительный анализ. Экспериментально подтверждены различия во временной эффективности последовательной программы и с использованием одного и более потоков: для любых размеров матриц не имеет смысла создавать параллельную реализацию для одного потока, так как для этого требуются дополнительные временные затраты на его создание. Для небольших размеров матриц порядка 500, 1000 наилучшее время среди остальных вариаций была выполнена программа с использованием восьми потоков, что соответствует числу логических ядер данной машины. При размерах квадратной матрицы, больше тысячи, с увеличением числа потоков, начиная с восьми, увеличивается скорость выполнения программы на 7\%.