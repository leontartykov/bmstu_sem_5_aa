\chapter{Аналитический раздел}

\section{Постановка задачи} 
\par
\textbf{Конвейер} – машина непрерывного транспорта \cite{mednov}, предназначенная для перемещения сыпучих, кусковых или штучных грузов.
\newline
\textbf{Конвейерное производство} - система поточной организации производства на основе конвейера, при которой оно разделено на простейшие короткие операции, а перемещение деталей осуществляется автоматически. Это такая организация выполнения операций над объектами, при которой весь процесс воздействия разделяется на последовательность стадий с целью повышения производительности путём одновременного независимого выполнения операций над несколькими объектами, проходящими различные стадии. Конвейером также называют средство продвижения объектов между стадиями при такой организации\cite{wiki}. Появилось в 1914 году на производстве Модели-Т на заводе Генри Форда\cite{ford} и произвело революцию сначала в автомобилестроении, а потом и во всей промышленности.

\section{Параллельное программирование}
\textbf{Параллельные вычисления} — способ организации компьютерных вычислений, при котором программы разрабатываются как набор взаимодействующих вычислительных процессов, работающих параллельно (одновременно). 

При использовании многопроцессорных вычислительных систем с общей памятью обычно предполагается, что имеющиеся в составе системы процессоры обладают равной производительностью, являются равноправными при доступе к общей памяти, и время доступа к памяти является одинаковым (при одновременном доступе нескольких процессоров к одному и тому же элементу памяти очередность и синхронизация доступа обеспечивается на аппаратном уровне). Многопроцессорные системы подобного типа обычно именуются симметричными мультипроцессорами (symmetric multiprocessors, SMP).

Перечисленному выше набору предположений удовлетворяют также активно развиваемые в последнее время многоядерные процессоры, в которых каждое ядро представляет практически независимо функциони рующее вычислительное устройство.

Обычный подход при организации вычислений для многопроцессорных вычислительных систем с общей памятью – создание новых параллельных методов на основе обычных последовательных программ, в которых или автоматически компилятором, или непосредственно программистом выделяются участки независимых друг от друга вычислений. Возможности автоматического анализа программ для порождения параллельных вычислений достаточно ограничены, и второй подход является преобладающим. При этом для разработки параллельных программ могут применяться как новые алгоритмические языки, ориентированные на параллельное программирование, так и уже имеющиеся языки, расширенные некоторым набором операторов для параллельных вычислений.

Широко используемый подход состоит и в применении тех или иных библиотек, обеспечивающих определенный программный интерфейс (application programming interface, API) для разработки параллельных программ. В рамках такого подхода наиболее известны Windows Thread API. Однако первый способ применим только для ОС семейства Microsoft Windows, а второй вариант API является достаточно трудоемким для использования и имеет низкоуровневый характер \cite{Barkalov}.

\subsection{Организация взаимодействия параллельных потоков}
Потоки исполняются в общем адресном пространстве параллельной программы. Как результат, взаимодействие параллельных потоков можно организовать через использование общих данных, являющихся доступными для всех потоков. Наиболее простая ситуация состоит в использовании общих данных только для чтения. В случае же, когда общие данные могут изменяться несколькими потоками, необходимы специальные усилия для организации правильного взаимодействия.

\section{Организация обработки}
У каждой линии конвейера есть очередь элементов. Когда линия еще активна, но элементов в очереди нет, линия уходит в режим ожидания.
По прошествию заданного времени линия проверяет не появились ли новые элементы в очереди. Если очередь не пустая, то нужно получить и обработать элемент, передать его следующей линии, если такая существует.

\section{Вывод}
В данном разделе были рассмотрены основы конвейерной обработки, технология параллельного программирования и организация взаимодействия параллельных потоков.