\chapter{Технологический раздел}
В данном разделе будут рассмотрены средства реализации и представлен листинг кода.

\section{Требования к програмному обеспечению}
Требования к программе при параллельной обработке:
\begin{itemize}
	\item объекты должны последовательно проходить конвейеры в заданном подядке;
	\item конвейеры должны работать каждый в своем потоке;
	\item конвейер должен завершать свою работу при поступлении специального элемента;
	\item до завершения работы конвейер должен ожидать поступления новых элементов.
\end{itemize}

\section{Средства реализации}
В качестве языка программирования был выбран C++ т.к. имеется опыт разработки на нем. Важной особенностью языка С++ является его высокая вычислительная производительность наряду с другими ЯП и возможность использования технологии параллельного программирования.\cite{c++}.
Среда разработки – QtCreator, которая предоставляет умную проверку кода, быстрое выявление ошибок и оперативное исправление, вместе с автоматическим рефакторингом кода и богатыми возможностями в навигации\cite{qt}.
Время работы было замерено с помощью функции clock() из библиотеки ctime. [2]
Для тестирования использовался компьютер на базе процессора Intel(R) Core(TM) i3, 4 логических ядра.

\section{Листинг кода алгоритмов}
В данном пункте представлен листинг кода функций. В листинге 1, 2, 3 представлен код рабочих потоков. В листинге 4 представлен код главного потока.
\begin{center}
	\captionsetup{justification=raggedright,singlelinecheck=off}
	\begin{lstlisting}[caption = Реализация первого уровня конвейера] 
void conveyor1(){
	int num = 0;
	while (1)
	{
		if (num == n)
		break;
		m1.lock();
		if (queue1.empty())
		{
			m1.unlock();
			continue;
		}
		input_t myObj = queue1.front();
		queue1.pop();
		m1.unlock();
		input_t newObj = caesar(myObj);
		m2.lock();
		queue2.push(newObj);
		sleep(1);
		clock_t time = clock();
		log.print(1, newObj, num, time);
		m2.unlock();
		num++;
	}
}
	\end{lstlisting}
\end{center}
\newpage
\begin{center}
	\captionsetup{justification=raggedright,singlelinecheck=off}
	\begin{lstlisting}[label=brute,caption = Реализация второго уровня конвейера]
void conveyor2(){
	int num = 0;
	while (1)
	{
		if (num == n)
		break;
		m2.lock();
		if (queue2.empty())
		{
			m2.unlock();
			continue;
		}
		input_t myObj = queue2.front();
		queue2.pop();
		m2.unlock();
		input_t newObj = upper_lower(myObj);
		m3.lock();
		queue3.push(newObj);
		sleep(3);
		clock_t time = clock();
		log.print(2, newObj, num, time);
		m3.unlock();
		num++;
	}
}
	\end{lstlisting}
\end{center}
\newpage
\begin{center}
	\captionsetup{justification=raggedright,singlelinecheck=off}
	\begin{lstlisting}[label=brute,caption = Реализация третьего уровня конвейера]
void conveyor3(){
	int num = 0;
	while (1)
	{
		if (num == n)
		break;
		m3.lock();
		if (queue3.empty())
		{
			m3.unlock();
			continue;
		}
		input_t myObj = queue3.front();
		queue3.pop();
		m3.unlock();
		input_t newObj = reverse(myObj);
		resm.lock();
		res.push_back(newObj);
		sleep(1.5);
		clock_t time = clock();
		log.print(3, newObj, num, time);
		resm.unlock();
		num++;
	}
}
	\end{lstlisting}
\end{center}
\newpage
\begin{center}
	\captionsetup{justification=raggedright,singlelinecheck=off}
	\begin{lstlisting}[label=brute,caption = Основная функция программы] 
int main(){
	f = fopen("log.txt", "w");
	n = 5;
	objvec.resize(n);
	
	thread t1(conveyor1);
	thread t2(conveyor2);
	thread t3(conveyor3);
	main_time = clock();
	
	for (int i = 0; i < n; i++)
	{
		string s = generate();
		objvec[i] = (s);
	}
	
	for (int i = 0; i < n; i++)
	{
		clock_t tm = clock();
		log.print(0, objvec[i], i, tm);
		m1.lock();
		
		queue1.push(objvec[i]);
		m1.unlock();
		sleep(2);
	}
	
	t1.join();
	t2.join();
	t3.join();
	fclose(f);
	printf("That's all folks!");
	return 0;
}
	\end{lstlisting}
\end{center}
\section*{Вывод}
\addcontentsline{toc}{section}{Вывод}
В данном разделе были рассмотрены основные сведения о модулях программы и листинг кода алгоритмов. 
