\chapter*{Введение}
\addcontentsline{toc}{chapter}{Введение}
Пути возможного решения NP-полных задач интересуют математиков со всего мира уже не одно десятилетие. Они пытаются понять, можно ли решить такие задачи за полиномиальное время. Одним из таких решений является использование эвристических алгоритмов, которые не гарантируют правильность решения задачи, но при верном подборе параметров позволяют обеспечить хорошее качество решения.

В последние два десятилетия при оптимизации сложных систем для поиска наилучших решений исследователи стали применять природные механизмы. Одним из предложенных решений таких задач является практическое применение муравьиных алгоритмов, которые основываются на поведении колонии муравьев\cite{ant_1}. Такой природных механизм позволяет решить такие задачи, как 
\grqqзадача коммивояжера\grqq, а также решения аналогичных задач поиска маршрутов на графах.  

Целью лабораторной работы является изучение предложенной оптимизации и подбор параметров наилучшего решения для двух классов данных. Для её достижения необходимо выполнить следующие задачи:
\begin{itemize}
	\item исследовать подходы алгоритмов полного перебора и оптимизации подражанием муравьиной колонии для решения задачи коммивояжера;
	\item привести схемы этих алгоритмов;
	\item описать классы эквивалентности тестов, необходимые для параметризации алгоритма муравьев;
	\item выполнить параметризацию муравьиного алгоритма на основе разработанных классов эквивалентности;
	\item провести сравнительный анализ этих алгоритмов на качество решения задачи и времени выполнения.
\end{itemize}