\chapter{Аналитический раздел}
В данном разделе описывается задача коммивояжера. Сформулированы пути решения методом "грубой силы" и муравьиной колонии. Описана их математическая модель. 

\section{Задача коммивояжёра}
Коммивояжёр (фр. commis voyageur) — бродячий торговец.

Является важной задачей, относящейся к транспортной логистике. Она формулируется следующим образом: "Рассматриваются N городов и попарное расстояние между каждым из них. Требуется найти такой порядок посещения городов, чтобы суммарное пройденное расстояние было минимальным, а каждый город посещался ровно один раз, при этом коммивояжер должен вернуться в тот город, откуда начал свой маршрут".

Математическую модель данной задачи можно описать следующим образом:
\begin{itemize}
	\item города и связь между ними можно представить в виде графа. Таким образом, вершины графа соответствуют городам, а ребра между вершинами - путям между сообщениями. 
\end{itemize}

В общем случае для гарантии существования маршрута принято считать, что граф является полностью связным - между произвольной парой вершин существует ребро.


\section{Алгоритм полного перебора}
Данную задачу гарантированно можно решить полным перебором (методом "грубой силы") всех возможных вариантов и последующим выбором оптимального. Для каждого такого варианта строится свой маршрут и определяется его длина. Происходит сравнение значения с локальным минимумом и вычисляется кратчайший. 

При таком подходе это число различных, неповторяющихся комбинаций зависит от общего числа городов $n$ и составляет $n!$, а значит время, затрачиваемое на полный перебор, растёт экспоненциально. При числе вершин графа, большем 10, вычисления могут занят часы, годы или даже столетия.

\section{Алгоритм муравьев}
\subsection{Принципы поведения муравьев}
Подход к оптимизации решения путем полного перебора основывается на принципе поведения муравьев. В этой основе лежит самоорганизация, которая позволяет обеспечить достижение общей цели колонии на уровне низкоуровневого взаимодействия. Важной особенностью такого взаимодействия является использование только локальной информации; исключается централизованное управление. В основе самоорганизации лежат результаты взаимодействия следующих компонентов:
\begin{itemize}
	\item случайность;
	\item многократность;
	\item положительная обратная связь;
	\item отрицательная обратная связь - испарение феромона.
\end{itemize}

Для передачи информации муравьи могут использовать непрямой способ обмена - стигмержи. Это разнесенный во времени тип взаимодействия, когда один субъект взаимодействия изменяет некоторую часть окружающей среды, а остальные используют информацию об ее состоянии позже, когда находятся в ее окрестности. Биологически стигмержи осуществляется через феромон - является достаточно стойким веществом; чем выше концентрация его концентрация на тропе, тем больше муравьев будет двигаться по ней. Со временем феромон испаряется, что позволяет адаптировать свое поведение под изменения внешней среды.

\subsection{Муравьиный подход к решению задачи}
Моделирование поведения муравьев связано с распределением феромона на тропе — ребре графа в задаче коммивояжера\cite{ant_2}. При этом вероятность включения ребра в маршрут отдельного муравья пропорциональна количеству феромона на этом ребре, а количество откладываемого феромона пропорционально длине маршрута. Чем короче маршрут, тем больше феромона будет отложено на его ребрах, следовательно, большее количество муравьев будет включать его в синтез собственных маршрутов. Моделирование такого подхода, использующего только положительную обратную связь, приводит к преждевременной сходимости — большинство муравьев двигается по локально оптимальному маршруту. Избежать, этого можно, моделируя отрицательную обратную связь в виде испарения феромона. При этом если феромон испаряется быстро, то это приводит к потере памяти колонии и забыванию хороших решений, с другой стороны, большое время испарения может привести к получению устойчивого локально оптимального решения. Теперь, с учетом особенностей задачи коммивояжера, мы можем описать локальные правила поведения муравьев при выборе пути.

\begin{itemize}
	\item муравьи имеют собственную «память». Поскольку каждый город может быть посещеи только один раз, у каждого муравья есть список уже посещенных городов --- список запретов. Обозначим через $J_{i,k}$ список городов, которые необходимо посетить муравью $k$, находящемуся в городе $i$;
	\item муравьи обладают «зрением» --- видимость есть эвристическое желание посетить город $j$, если муравей находится в городе $i$. Будем считать, что видимость обратно пропорциональна расстоянию между городами $i$ и $j$ --- $D_{ij}$ по формуле (\ref{eq:vision}):
	\begin{equation}
		\label{eq:vision}
		\eta_{ij} = \frac{1}{D_{ij}}
	\end{equation}
	\item муравьи обладают «обонянием» — они могут улавливать след феромона, подтверждающий желание посетить город $j$ из города $i$, на основании опыта других муравьев. Количество феромона на ребре $(i,j)$ в момент времени $t$ обозначим через $\tau_{ij}(t)$.
\end{itemize}

На этом формулируется вероятностно-пропорциональное правило, определяющее вероятность перехода $k$-ого муравья из города $i$ в город $j$ по формуле (\ref{eq:rule}):

\begin{equation}
	\label{eq:rule}
	P_{i,j,k}(t) =
	\begin{cases}
		 \frac{[\tau_{ij}(t)]^\alpha*[\eta_{ij}]^\beta}{\sum_{l\in J_{i,k}}^{}[\tau_{il}(t)]^\alpha * [\eta_{il}]^\beta}, & j \in J_{i,k};\\
		0, & j \notin J_{i,k},
	\end{cases}
\end{equation}


где $\alpha, \beta$ — параметры, задающие веса следа феромона, при $\alpha=0$ алгоритм вырождается до жадного алгоритма (будет выбран ближайший город). Выбор города является вероятностным, правило \ref{eq:rule} лишь определяет ширину
зоны города $j$; в общую зону всех городов $J_{i,k}$;, бросается случайное число, которое и определяет выбор муравья. Правило по формуле \ref{eq:rule} не изменяется в ходе алгоритма, но у двух разных муравьев значение вероятности перехода будут отличаться, т. к. они имеют разный список разрешенных городов.

Пройдя ребро $(i,j)$, муравей откладывает на нем некоторое количество феромона, которое должно быть связано с оптимальностью сделанного выбора. Пусть $\tau_k(t)$ есть маршрут, пройденный муравьем $k$ к моменту времени t, а $L_k(t)$ --- длина этого маршрута. Пусть также $Q$ --- параметр, имеющий значение порядка длины оптимального пути. Тогда откладываемое количество феромона может быть задано по формуле (\ref{eq:pheromone_drop}):

\begin{equation}
	\label{eq:pheromone_drop}
	\Delta\tau_{i,j,k}(t) =
	\begin{cases}
		\frac{Q}{L_{k}(t)}, & (i,j) \in T_{k}(t);\\
		0, & (i,j) \notin T_{k}(t).
	\end{cases}
\end{equation}

Правила внешней среды определяют, в первую очередь, испарение феромона. Пусть $\rho \in [0,1]$ есть коэффициент испарения, тогда правило испарения имеет вид по формуле (\ref{eq:pheromone_evaporation}):

\begin{equation}
	\label{eq:pheromone_evaporation}
	\tau_{ij}(t+1) = (1 - \rho) * \tau_{ij}(t) + \Delta\tau_{ij}(t); \Delta\tau_{ij}(t) = \sum_{k = 1}^{m} \Delta\tau_{ij,k}(t); 
\end{equation}

где $m$ — количество муравьев в колонии.

В начале алгоритма количество феромона на ребрах принимается равным
небольшому положительному числу. Общее количество муравьев остается постоянным и равным количеству городов, каждый муравей начинает маршрут из своего города.
Подбор таких параметров, при которых алгоритм выдает верное решение называется параметризацией.

\section{Вывод}
В данном разделе была описана задача коммивояжера. Сформулированы пути решения методом "грубой силы" и муравьиной колонии. Описана их математическая модель.

Разрабатываемое программное обеспечение должно реализовывать эти два подхода к решению задачи. На вход программе подается симметричная матрица смежностей, а также регулируемые параметры. Программа должна выдавать результат - длину кратчайшего пути и способ обхода всех городов.