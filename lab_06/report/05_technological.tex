\chapter{Технологический раздел}
В данном разделе приведены листинги реализованных алгоритмов, требование к программному обеспечению и его тестирование.

\section{Требования к программному обеспечению}
Программа должна удовлетворять следующим требованиям:
\begin{itemize}
	\item на вход программе подается симметричная матрица стоимостей графа, количество дней жизни колонии, а также регулируемые параметры;
	\item на вход программе подаются корректные данные;
	\item на вход программа выдает массив кратчайшего пути и его длину этого пути;
\end{itemize}


\section{Выбор средств реализации}
Для реализации алгоритмов в данной лабораторной работе был выбран язык программирования C++ \cite{c++}. Данный выбор обусловен тем, что он является компилируемым, что увеличивает скорость выполнения программ по сравнению с интерпретируемыми. В качестве среды разработки был использован Visual Studio Code\cite{vs}, так как в нем присутствуют инструменты для удобства написания и отладки кода. Выполнение замера процессорного времени осуществлено при помощи библиотеки chrono \cite{chrono}.

\section{Листинги программ}
\subsection{Полный перебор}
Ниже представлены листинги разработанного модуля метода полного перебора.
\begin{lstlisting}[label = brute_force, caption=Программный код алгоритма полного перебора.]
int find_short_way_by_brute_force(array_t *cities, int **matrix)
{
	output_cities(cities);
	int short_length = 0, min_short_length = 0;
	int i = 0, min_i = 0;
	
	short_length = find_way(cities, matrix);
	min_short_length = short_length;
	i++;
	
	while (find_next_permutation(cities) == true)
	{
		output_cities(cities);
		short_length = find_way(cities, matrix);
		if (short_length < min_short_length){
			min_short_length = short_length;
			min_i = i;
		}
		
		i++;
	}
	
	return min_short_length;
}
\end{lstlisting}

\subsection{Муравьиный алгоритм}
На листинге представлен разработанного модуль муравьиного алгоритма:
\newpage
\begin{lstlisting}[label=ant_algorithm,caption=Программный код муравьиного алгоритма.]
void find_short_way_by_ant_algorithm(short_route_t &shortest_route, matrix_int_t *path,
ant_colony_t *colony)
{
	int ant_route_length = 0;
	
	(*colony).meta_data_colony.Q = calculate_Q(path);
	
	int t_max = (*colony).t_max;
	int count_ants = (*path).size_row;
	
	generate_start_city_for_ants(colony->ants, count_ants);
	int short_ant_route = (*colony).meta_data_colony.Q*2;
	int index_last_city = 0;
	
	for (int t = 0; t < t_max; t++)
	{
		for (int i = 0; i < count_ants; i++)
		{
			index_last_city = i;
			ant_route_length = get_route_from_ant(&colony->ants[i], &colony->pheromon,
			&colony->attractiveness, &colony->meta_data_colony, path,
			index_last_city);
			
			if (ant_route_length < short_ant_route)
			{
				shortest_route.length_short_route = ant_route_length;
				copy_ant_route_to_short(shortest_route, colony->ants[i].visit_cities);
			}            	
		}

		add_pheromone_route(colony, (*path).size_row);
		evaporate_pheromon(&colony->pheromon, colony->meta_data_colony.koeff_evaporation,
		colony->meta_data_colony.start_evaporation);
		reset_ant_to_default(&colony->ants[i]);
	}
}
\end{lstlisting}

\section{Вспомогательные модули}
\subsection{Полный перебор}
На листингах представлены вспомогательные программные модули, которые используются в алгоритме полного перебора:
\begin{lstlisting}[label=permutation,caption=Программный код нахождения следующей перестановки.]
bool find_next_permutation(array_t *cities)
{
	int j = (*cities).size_array - 2;
	while (j != -1 && (*cities).array[j] >= (*cities).array[j + 1]){
		j--;
	}
	
	if (j == -1)
	return false;
	
	int k = (*cities).size_array - 1;
	while ((*cities).array[j] >= (*cities).array[k]){
		k--;
	}
	
	swap_elems_array(cities->array, j, k);
	int left = j + 1, right = (*cities).size_array - 1;
	
	while (left < right){
		swap_elems_array(cities->array, left++, right--);
	}
	
	return true;
}
\end{lstlisting}
\subsection{Муравьиный алгоритм}
На листингах представлены вспомогательные программные модули, которые используются в муравьином алгоритме:
\newpage
\begin{lstlisting}[label=Q_code, caption=Программный код вычисления параметра Q.]	
int calculate_Q(matrix_int_t *path)
{
	int size_row = (*path).size_row;
	int size_column = (*path).size_column;
	int average_length = 0;
	
	for (int i = 0; i < size_row; i++){
		for (int j = 0; j < i; j++){
			average_length += (*path).matrix[i][j];
		}
	}
	
	average_length /= 2;
	return average_length;
}
\end{lstlisting}

\begin{lstlisting}[label=get_route, caption=Программный код вычисления маршрута каждого муравья.]	
int get_route_from_ant(ant_t *ant, matrix_double_t *pheromons,
matrix_double_t *attractivness, meta_t *meta_data_colony,
matrix_int_t *paths, int index_last_city)
{
	int chosen_city;
	int count_cities = ant->visit_cities->size_array;
	double get_way_length = 0;

	(*ant).count_visited_cities += 1;
	int last_index_last_city = index_last_city;
	
	for (int i = 0; i < count_cities - 1; i++)
	{
		chosen_city = choose_next_city(&get_way_length, ant, paths, pheromons,
		attractivness, last_index_last_city, *meta_data_colony);
		last_index_last_city = chosen_city;
		(*ant).visit_cities->array[i+1] = chosen_city;
		(*ant).count_visited_cities += 1;
		(*ant).visited_way_length += get_way_length;
	}

	int last_visited_city = (*ant).visit_cities->array[count_cities - 1];

	(*ant).visited_way_length += paths->matrix[last_index_last_city][index_last_city];
	return (*ant).visited_way_length;
}
\end{lstlisting}

\begin{lstlisting}[label=next_city, caption=Программный код выбора муравьем следующего города.]
int choose_next_city(double *get_way_length, ant_t *ant, matrix_int_t *paths,
matrix_double_t *pheromons, matrix_double_t *attractiveness,
int index_last_city, meta_t meta_data_colony)
{
	int count_cities = (*ant).visit_cities->size_array;
	double denominator = 0, sum = 0;
	int chosen_next_city;

	for (int i = 0; i < count_cities; i++){
		if (check_is_visited_city(i, ant->visit_cities, ant->count_visited_cities) == false)
		{
			sum = pow(pheromons->matrix[index_last_city][i], meta_data_colony.alpha) +
			pow(attractiveness->matrix[index_last_city][i], meta_data_colony.beta);
			denominator += sum;
		}
	}
	
	double *probability = new double[count_cities]{};
	for (int i = 0; i < count_cities; i++)
	{
		if (check_is_visited_city(i, ant->visit_cities, ant->count_visited_cities) == false)
		{
			sum = pow(pheromons->matrix[index_last_city][i], meta_data_colony.alpha) +
			pow(attractiveness->matrix[index_last_city][i], meta_data_colony.beta);
			probability[i] = sum / denominator;
		}
	}
	
	chosen_next_city = get_value_from_fortuna_wheel(probability, count_cities);
	
	*get_way_length = paths->matrix[index_last_city][chosen_next_city];
	return chosen_next_city;
}
\end{lstlisting}

\begin{lstlisting}[label=fortuna_wheel, caption=Программный код выбора города случайным образом ("колесо фортуны").]
int get_value_from_fortuna_wheel(double *probability, int count_cities)
{
	srand(static_cast<unsigned int>(time(0)));
	double generated_probability = (double)(rand())/RAND_MAX;
	
	double sum = 0;
	bool is_index_city_find = false;
	int find_index = 0;
	for (int i = 0; i < count_cities && is_index_city_find == false; i++){
		sum += probability[i];
		if (sum < generated_probability){
			find_index++;
		}
		else{
			find_index = i;
			is_index_city_find = true;
		}
	}
	
	return find_index; 
	
}
\end{lstlisting}

\section{Тестирование}
Для тестирования используется метод черного ящика. В данном разделе приведена таблица \ref{table:ref1}, в которой указаны тесты:

\begin{table}[ht!]
	\centering
	\captionsetup{singlelinecheck = false, justification=raggedright}
	\caption{Таблица тестов}
	\label{table:ref1}
	\begin{tabular}{|c|c|c|}
		\hline
		№ & Матрица  &    Ожидаемый результат\\\hline
		1 & $\begin{pmatrix}
			0 & 5 & 4 & 8 & 3 \\
			5 & 0 & 6 & 7 & 9 \\
			4 & 6 & 0 & 5 & 5 \\
			8 & 7 & 5 & 0 & 4 \\
			3 & 9 & 5 & 4 & 0 
			\end{pmatrix}$				 & 30
		\\ \hline
		2 & $\begin{pmatrix}
			0 & 6 & 8 & 1 \\
			6 & 0 & 3 & 5 \\
			8 & 3 & 0 & 9 \\
			1 & 5 & 9 & 0 \\
		\end{pmatrix}$	 				& 17
		\\ \hline
	\end{tabular}
\end{table}
	Для метода "грубой силы" и муравьиного алгоритма данные тесты пройдены успешно.
	
\section{Вывод}
В данном разделе был выбран язык программирования, среда разработки, приведены листинги модулей. Было успешно проведено их тестирование методом черного ящика по таблице \ref{table:ref1}. 

