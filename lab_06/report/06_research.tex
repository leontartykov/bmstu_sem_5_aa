\chapter{Исследовательский раздел}
В данном разделе приведены технические характеристики устройства; классы данных, на которых были проведены эксперименты; результаты параметризации и выборка из нее наилучших результатов.
 
\section{Технические характеристики}
Технические характеристики устройства, на котором выполнялось тестирование:
\begin{itemize}
	\item операционная система: Ubuntu 21.10;
	\item память: 8 GiB;
	\item процессор: Intel(R) Core(TM) i5-8265U CPU @ 1.60GHz   1.80 GHz.
\end{itemize}
Тестирование проводилось на ноутбуке, который был подключен к сети питания. Во время проведения тестирования ноутбук был нагружен только встроенными приложениями окружения, самим окружением и системой тестирования.

\section{Постановка эксперимента}
В муравьином алгоритме вычисления производятся на основе настраиваемых параметров. Рассмотрим два класса данных и выполним подбор параметров, при которых метод даст точный результат.

Будут рассматриваться матрицы размерности 10$\times$10. Это необходимо для того, чтобы результат зависел от параметров.

В качестве первого класса данных будет является матрица стоимостей, в которой значения варьируются в диапазоне [5; 10]; во втором классе - значения лежат в диапазоне [50; 100].

Будем запускать муравьиный алгоритм для всех значений $\alpha$, $\beta$, $\rho \in [0,1]$ с шагом $= 0.1$, пока не будет найдено точное значение или приближенное для каждого набора.

В результате тестирования будет выведена таблица со значениями $\alpha, \beta, \rho, Длина, Разница$, где $Длина$ — оптимальная длина пути, найденная муравьиным алгоритмом, $Разница$ - Разность между полученным значением и настоящей оптимальной длиной пути, а $\alpha, \beta, \rho$ — настраиваемые параметры.

\subsection{Класс данных №1}
Класс данных №1 описан следующей матрицей стоимостей (\ref{eq:matrix_1}):
\begin{equation}
	\label{eq:matrix_1}
	M = \begin{pmatrix}
		0  & 3 & 5  & 9 & 10 & 8  & 7  & 6 & 4  & 5 \\
		3  & 0 & 6  & 8 & 9  & 5  & 9  & 8 & 7  & 7 \\
		5  & 6 & 0  & 5 & 10 & 10 & 8  & 6 & 4  & 7 \\
		9  & 8 & 5  & 0 & 3  & 3  & 4  & 5 & 5  & 6 \\
		10 & 9 & 10 & 3 & 0  & 8  & 9  & 7 & 10 & 5 \\
		8  & 5 & 10 & 3 & 8  & 0  & 10 & 8 & 3  & 7 \\
		7  & 9 & 8  & 4 & 9  & 10 & 0  & 5 & 6  & 9 \\
		6  & 8 & 6  & 5 & 7  & 8  & 5  & 0 & 8  & 8 \\
		4  & 7 & 4  & 5 & 10 & 3  & 6  & 8 & 0  & 3 \\ 
		5  & 7 & 7  & 6 & 5  & 7  & 9  & 8 & 3  & 0
	\end{pmatrix}
\end{equation}
Результаты работы алгоритма с различными комбинация параметров представлены в приложении А. Таблица \ref{table:result_1} содержит выборку параметров, которые помогают решить задачу наилучшим образом.

\begin{table}[ht!]
	\centering
	\captionsetup{singlelinecheck = false, justification=raggedright}
	\caption{Выборка параметров}
	\label{table:result_1}
	\begin{tabular}{|c|c|c|c|c|}
		\hline	
		$\alpha$        & $\beta$      & $\rho$      &Длина  & Разница \\
		\hline
		0.1  & 0.9  & 0.8  & 42    & 0     \\
		0.2  & 0.8  & 0.7  & 42    & 0     \\
		0.3  & 0.7  & 0.9  & 42    & 0     \\
		0.3  & 0.7  & 1    & 42    & 0     \\
		0.4  & 0.6  & 0    & 42    & 0     \\
		0.8  & 0.2  & 0.5  & 42    & 0     \\
		0.9  & 0.1  & 0.6  & 42    & 0     \\
		0.7  & 0.3  & 0.9  & 42    & 0     \\
		\hline
	\end{tabular}
\end{table}

\subsection{Класс данных №2}
Класс данных №1 описан следующей матрицей стоимостей (\ref{eq:matrix_2}):
\begin{equation}
	\label{eq:matrix_2}
	M = \begin{pmatrix}
		0  & 53 & 65 & 79 & 90 & 82 & 76 & 61 & 47 & 55 \\ 
		53 & 0  & 62 & 84 & 95 & 53 & 91 & 68 & 77 & 76 \\
		65 & 62 & 0  & 50 & 99 & 73 & 58 & 61 & 64 & 57 \\
		79 & 84 & 50 &  0 & 93 & 63 & 54 & 55 & 65 & 86 \\
		90 & 95 & 99 & 93 &  0 & 98 & 92 & 67 & 73 & 54 \\
		82 & 53 & 73 & 63 & 98 &  0 & 82 & 81 & 73 & 70 \\
		76 & 91 & 58 & 54 & 92 & 82 &  0 & 50 & 60 & 90 \\
		61 & 68 & 61 & 55 & 67 & 81 & 50 &  0 & 81 & 82 \\
		47 & 77 & 64 & 65 & 73 & 73 & 60 & 81 &  0 & 63 \\
		55 & 76 & 57 & 86 & 70 & 70 & 90 & 82 & 63 &  0
	\end{pmatrix}
\end{equation}
Результаты работы алгоритма с различными комбинация параметров представлены в приложении А. Таблица \ref{table:result_2} содержит выборку параметров, которые помогают решить задачу наилучшим образом.
\begin{table}[ht!]
	\centering
	\captionsetup{singlelinecheck = false, justification=raggedright}
	\caption{Выборка параметров}
	\label{table:result_2}
	\begin{tabular}{|c|c|c|c|c|}
		\hline	
		$\alpha$        & $\beta$      & $\rho$     &Длина  & Разница \\
		\hline
		0.1  & 0.9  & 0.6  & 554    & 0     \\
		0.2  & 0.8  & 0.2  & 554    & 0     \\
		0.3  & 0.7  & 0.4  & 554    & 0     \\
		0.4  & 0.6  & 0.7  & 554    & 0     \\
		0.4  & 0.6  & 0.9  & 554    & 0     \\
		0.8  & 0.2  & 0.2  & 554    & 0    \\
		0.9  & 0.1  & 0.5  & 554    & 0     \\
		0.6  & 0.4  & 0.8  & 554    & 0     \\
		\hline
	\end{tabular}
\end{table}

\section{Вывод}
Была выполнена параметризация алгоритма муравьев для двух классов данных. Данная оптимизация при наборе параметров $\alpha = 0.3, \beta = 0.7, \rho = 0.4$ показала увеличение скорости выполнения в 400 раз по сравнению с полным перебором.