\chapter*{Заключение}
\addcontentsline{toc}{chapter}{Заключение}
В данной лабораторной работе были рассмотрены основополагающие материалы, которые потребовались для решения задачи коммивояжера. Были разработаны классы эквивалентности, проведена для них параметризация алгоритма муравьев. Был проведен сравнительный анализ двух методов решения задачи: полный перебор и муравьиного.

Муравьиный алгоритм позволяет обеспечить качественное решение, насколько это возможно. Для этого необходимо подбирать параметры для разных входных данных. Данный метод позволяет решить задачу коммивояжера в 400 раз быстрее полного перебора, что позволяет использовать его в качестве оптимизации. Алгоритм "грубой силы" является универсальным, для него не нужно проводить параметризацию, выдает гарантированно точный результат, но за длительное время (на размерностях матрицы, больше 20, вычисления могут занимать несколько дней, столетий). Поэтому важным критерием выбора способа решения является скорость выполнения: если она - ключевой фактор, то необходимо использовать муравьиный алгоритм.