\chapter{Аналитическая часть}

В данном разделе представлены теоретические сведения о рассматриваемых алгоритмах поиска в словаре.

\section{Теоретические данные}
Словари - объекты, которые записываются парой "ключ-значение". Ключи в словаре должны быть уникальными. Поиск необходимой информации в словаре является одной из фундаментальной задачей программирования.

\section{Описание словаря}
В данной лабораторной работе использован отрывок из книги Шерлок Холмс. Словарь представляет собой ключ "слово в словаре", значение - частота появления этого слова в отрывке.

\section{Алгоритм полного перебора}
Алгоритмом полного перебора называют метод решения задачи, при котором по очереди рассматриваются все возможные варианты исходного набора данных. В случае словарей будет произведен последовательный перебор элементов словаря до тех пор, пока не будет найден необходимый. сложность такого алгоритма зависит от количества всех возможных решений, а время работы может стремиться к экспоненциальному.

Пусть алгоритм нашел элемент на первом сравнении. Тогда, в лучшем случае, будет затрачено $k_0 + k_1$ операций, на втором -- $k_0 + 2k_1$, на $N$ -- $k_0 + Nk_1$. тогда, средняя трудоемкость может быть рассчитано по формуле \eqref{eq:brute}, где $\Omega$ - множество всех возможных случаев.

\begin{equation}
	\label{eq:brute}
	\sum_{i \in \Omega} p_i t_i = (k_0 + k_1) \frac{1}{N + 1} + (k_0 + 2k_1) * \frac{1}{N + 1} + \dots + (k_0 + Nk_1) * \frac{1}{N + 1} 
\end{equation}

Из \eqref{eq:brute}, сгруппировав слагаемые, получим итоговую формулу для расчета средней трудоемкости работы алгоритма:

\begin{equation}
	k_0 + k1(\frac{N}{N + 1} + \frac{N}{2}) = k_0 + k1(1 + \frac{N}{2} - \frac{1}{N + 1}) 
\end{equation}

\section{Алгоритм двоичного поиска}
Бинарный поиск выполняется для отсортированных данных. Он позволяет сравнивать ключ со средним элементом словаря; если он меньше, то продолжается поиск в левой части, иначе - в правой.

Использование данного алгоритма в для поиска в словаре в любом из случаев будет иметь трудоемкость равную $O(log_2(N))$ \cite{knut}. Несмотря на то, что в среднем и худшем случаях данный алгоритм работает быстрее алгоритма полного перебора, стоит отметить, что предварительная сортировка больших данных требует дополнительных затрат по времени и может оказать серьезное действие на время работы алгоритма. Тем не менее, при многократном поиске по одному и тому же словарю, применение алгоритма сортировки понадобится всего один раз.

\section{Разбиение словаря на сегменты}
Предлагается разбить словарь на сегменты для повышения оптимизации поиска в нем. Критерием для разбиения выбирается первая буква слова. Когда найден сегмент, в котором лежит слово, выполняется бинарный поиск в это сегменте.

\section{Вывод}
В данной работе описана задача реализации поиска в словаре. Были рассмотрены алгоритмы реализации данного поиска.

Входными данными для программного обеспечения являются:
\begin{itemize}
	\item словарь из записей, вида \[\{word: string, value: int\}\] для поиска по нему;
	\item ключ для поиска в словаре.
\end{itemize}

Выходными данными является найденная в словаре запись для каждого из реализуемых алгоритмов из предложенного для пользователя меню.