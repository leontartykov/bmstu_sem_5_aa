\chapter{Технологический раздел}
В данном разделе приведены листинги реализованных алгоритмов, требование к программному обеспечению и его тестирование.

\section{Требования к программному обеспечению}
Программа должна удовлетворять следующим требованиям:
\begin{itemize}
	\item на вход программе подается имя файла, откуда считывается отрывок текста;
	\item на вход программе подаются корректные данные;
	\item на выход программа выдает найденное значение по ключу или сообщение об ошибке.
\end{itemize}


\section{Выбор средств реализации}
Для реализации алгоритмов в данной лабораторной работе был выбран язык программирования Python 3.9.7\cite{python3}. В качестве среды разработки был использован Visual Studio Code\cite{vs}, так как в нем присутствуют инструменты для удобства написания и отладки кода.

\section{Листинги программ}
Ниже представлены листинг разработанного модуля метода полного перебора.
\newpage
\begin{lstlisting}[label = full_search, caption=Программный код алгоритма полного перебора.]
def find_by_full_search(self, word):
	is_find_key = False
	value = 0
	count_compares = 0
	for key in self.dictionary.keys():
		if is_find_key == True:
			break
		if key == word:
			is_find_key = True
			value = self.dictionary[key]
		count_compares += 1
	return value, count_compares
\end{lstlisting}

Ниже представлены листинг разработанного модуля метода бинарного поиска.
\begin{lstlisting}[label = binary_search, caption=Программный код алгоритма полного перебора.]
def find_by_binary_search(self, word, dictionary=None):
	if dictionary == None:
		dictionary = self.sorted_dictionary
		left = 0; right = self.len - 1
	else:
		left = 0; right = len(dictionary) - 1

	is_find = False; value = 0
	list_dictionary = list(dictionary) 
	count_compares = 0
	while is_find == False and right >= left:
		middle = (left + right) // 2
		key = list_dictionary[middle]
		if key == word:
			is_find = True
			value = dictionary.get(key)
		elif key > word:
			right = middle - 1
		else:
			left = middle + 1
		count_compares += 1
	return value, count_compares
\end{lstlisting}
\newpage
Ниже представлены листинг разработанного модуля сегментного разбиения с бинарным поиском.
\begin{lstlisting}[label = segment_binary_search, caption=Программный код алгоритма сегментного разбиения с бинарным поиском.]
def find_by_segment_search(self, word):
	count_compares = 0
	count_segments = len(self.segment_dictionary)
	result = 0; count_companies = 0
	for i in range(count_segments):
		if word[0] == list(self.segment_dictionary[i])[0][0]:
			result, binary_search_compares = self.find_by_binary_search(word, self.segment_dictionary[i])
			count_compares += binary_search_compares
			break
	return result, count_compares
\end{lstlisting}

\section{Тестирование ПО}
Результаты тестирования ПО приведены в таблице \ref{table:testing}. 
\begin{table}[H]
	\captionsetup{singlelinecheck = false, justification=raggedleft}
	\caption{Тестирование ПО}
	\begin{center}
		\begin{tabular}{|c|c|c|}
			\hline
			Входные данные & Ожидаемый результат  & Результат \\ \hline
			and            & 331                  & 331       \\ \hline
			together       & 1                    & 1         \\ \hline
			1         	   & 2                    & 2         \\ \hline
			wednesday?     & not found            & not found \\ \hline
		\end{tabular}
	\end{center}
	\label{table:testing}
\end{table}

\section{Вывод}
Было написано и протестировано программное обеспечение для решения поставленной задачи.