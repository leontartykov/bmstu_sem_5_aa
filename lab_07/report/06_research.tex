\chapter{Исследовательский раздел}
В данном разделе приведены технические характеристики устройства; классы данных, на которых были проведены эксперименты; результаты параметризации и выборка из нее наилучших результатов.
 
\section{Технические характеристики}
Технические характеристики устройства, на котором выполнялось тестирование:
\begin{itemize}
	\item операционная система: Ubuntu 21.10;
	\item память: 8 GiB;
	\item процессор: Intel(R) Core(TM) i5-8265U CPU @ 1.60GHz   1.80 GHz.
\end{itemize}
Тестирование проводилось на ноутбуке, который был подключен к сети питания. Во время проведения тестирования ноутбук был нагружен только встроенными приложениями окружения, самим окружением и системой тестирования.

\section{Постановка эксперимента}
Эксперимент проведен на данных типа "строка". Количество элементов в словаре фиксировано и равно 2241. Проведенный эксперимент устанавливает зависимость количество сравнений при поиске от позиции элемента в словаре.

Во время тестирования устройство было подключено к блоку питания и не нагружено никакими приложениями, кроме встроенных приложений окружения, окружением и системой тестирования. Оптимизация компилятора была отключена.

\section{Результаты эксперимента}
Результаты эксперимента приведены в приложениях А, Б, В для полного перебора, бинарного поиска и бинарного поиска в сегментированном словаре соответственно. 

В среднем, при поиске полным перебором для каждого ключа осуществляется в 131 раз больше сравнений, чем для бинарного поиска и в 206 раз больше сравнений, чем для поиска полным перебором. 

\section{Вывод}\label{sec:exp-sum}
Качественная оценка работы алгоритма зависит от количества сравнений с ключами при поиске. В среднем, при поиске полным перебором для каждого ключа осуществляется в 131 раз больше сравнений, чем для бинарного поиска и в 206 раз больше сравнений, чем для поиска полным перебором. Исходя из результатов эксперимента, можно сделать вывод, что самым оптимальным алгоритмом поиска из трех предложенных является алгоритм бинарного поиска с предварительной сегментацией словаря.