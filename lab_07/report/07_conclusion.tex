\chapter*{Заключение}
\addcontentsline{toc}{chapter}{Заключение}
В данной лабораторной работе были рассмотрены основополагающие материалы, которые потребовались для решения задачи поиска в словаре полным перебором, бинарного поиска и бинарного поиска с предварительной сегментацией словаря. Однако, он требует дополнительных вычислительных затрат на сегментацию словаря и дополнительный объем памяти на хранение выделенных сегментов, что отражает формула \ref{math:mem-seg}. Были разработано и реализовано программное обеспечение, проведены соответствующие эксперименты.

Опираясь на проведенное исследование, можно сделать вывод, что самым оптимальным подходом к поиску является разделение словаря на сегменты и осуществление бинарного поиска в каждом сегменте, особенно в случаях, когда словарь, подающийся на вход алгоритму, уже сегментирован.
